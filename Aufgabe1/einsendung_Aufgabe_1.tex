\documentclass[a4paper,10pt,ngerman]{scrartcl}
\usepackage{babel}
\usepackage[T1]{fontenc}
\usepackage[utf8x]{inputenc}
\usepackage[a4paper,margin=2.5cm,footskip=0.5cm]{geometry}

% Die nächsten vier Felder bitte anpassen:
\newcommand{\Aufgabe}{Aufgabe 1: Alice im Wunderland} % Aufgabennummer und Aufgabennamen angeben
\newcommand{\TeamId}{01093}                       % Team-ID aus dem PMS angeben
\newcommand{\TeamName}{Kimi and his Droogs}                 % Team-Namen angeben
\newcommand{\Namen}{Kimi Müller}           % Namen der Bearbeiter/-innen dieser Aufgabe angeben

% Kopf- und Fußzeilen
\usepackage{scrlayer-scrpage, lastpage}
\setkomafont{pageheadfoot}{\large\textrm}
\lohead{\Aufgabe}
\rohead{Team-ID: \TeamId}
\cfoot*{\thepage{}/\pageref{LastPage}}

% Position des Titels
\usepackage{titling}
\setlength{\droptitle}{-1.0cm}

% Für mathematische Befehle und Symbole
\usepackage{amsmath}
\usepackage{amssymb}

% Für Bilder
\usepackage{graphicx}

% Für Algorithmen
\usepackage{algpseudocode}

% Für Quelltext
\usepackage{listings}
\usepackage{color}
\definecolor{mygreen}{rgb}{0,0.6,0}
\definecolor{mygray}{rgb}{0.5,0.5,0.5}
\definecolor{mymauve}{rgb}{0.58,0,0.82}
\lstset{
  inputencoding=utf8x,
  extendedchars=true,
  keywordstyle=\color{blue},commentstyle=\color{mygreen},
  stringstyle=\color{mymauve},rulecolor=\color{black},
  basicstyle=\footnotesize\ttfamily,numberstyle=\tiny\color{mygray},
  captionpos=b, % sets the caption-position to bottom
  keepspaces=true, % keeps spaces in text
  numbers=left, numbersep=5pt, showspaces=false,showstringspaces=false,
  showtabs=false, stepnumber=1, tabsize=2, title=\lstname
}
\lstdefinelanguage{JavaScript}{ % JavaScript ist als einzige Sprache noch nicht vordefiniert
  keywords={break, case, catch, continue, debugger, default, delete, do, else, finally, for, function, if, in, instanceof, new, return, switch, this, throw, try, typeof, var, void, while, with},
  morecomment=[l]{//},
  morecomment=[s]{/*}{*/},
  morestring=[b]',
  morestring=[b]",
  sensitive=true
}

% Diese beiden Pakete müssen zuletzt geladen werden
%\usepackage{hyperref} % Anklickbare Links im Dokument
\usepackage{cleveref}

% Daten für die Titelseite
\title{\textbf{\Huge\Aufgabe}}
\author{\LARGE Team-ID: \LARGE \TeamId \\\\
	    \LARGE Team-Name: \LARGE \TeamName \\\\
	    \LARGE Bearbeiter/-innen dieser Aufgabe: \\
	    \LARGE \Namen\\\\}
\date{\LARGE\today}

\begin{document}

\maketitle
\tableofcontents

\vspace{0.5cm}

\textbf{Anleitung:} Trage oben in den Zeilen 8 bis 11 die Aufgabennummer, die Team-ID, den Team-Namen und alle Bearbeiter/-innen dieser Aufgabe mit Vor- und Nachnamen ein.
Vergiss nicht, auch den Aufgabennamen anzupassen (statt "`\LaTeX-Dokument"')!

Dann kannst du dieses Dokument mit deiner \LaTeX-Umgebung übersetzen.

Die Texte, die hier bereits stehen, geben ein paar Hinweise zur
Einsendung. Du solltest sie aber in deiner Einsendung wieder entfernen!

\section{Lösungsidee}
Abstrahiert ist das Lückenwort eine Buchstabenfolge beliebiger Länge >= 1. Daher suchen wir einfach nach dem Kontext des Lückenwortes, die Lücke wird da mit dieser Buchstabenfolge ersetzt.

\section{Umsetzung}
Mithilfe des regulären Ausdrucks "w+" für das Lückenwort suchen wir nach Übereinstimmungen. "w" ist eine Abkürzung für alle Wortbestandteile, so muss sich keine Sorge um Umlaute o.ä. gemacht werden. "+" steht für "1 oder mehrere Instanzen des vorherigen Worts"

\section{Beispiele}
Die inkludierten Beispiele:
\begin{lstlisting}[language=Python]
['das kommt mir gar nicht richtig vor']
['ich muß doch clara sein', 'ich muß in clara verwandelt']
['fressen spatzen gern katzen', 'fressen katzen gern spatzen']
['das spiel fing an', 'das publikum fing an']
['ein sehr schöner tag']
['wollen sie so gut sein']
\end{lstlisting}

Zeile 3 sticht als besonders hervor, wo zum ersten Mal unerwartetes Verhalten auftrat. Wird das Buch unabhängig von Zeilenabsätzen durchsucht, kommt 'fressen katzen gern spatzen' doppelt vor. Dem wird vorgebeugt, indem identische Treffer gelöscht.

\begin{lstlisting}[language=Python]
['alice in späterer zeit']
['aufschrift eingemachte']
\end{lstlisting}

Beispiel (1) ist relevant, da im Buch ein Teil des gesuchten innerhalb einer neuen Zeile ist. Daraus ergibt sich sowohl das Problem, dass nicht nach Leerzeichen gematcht werden muss, sondern alle "leeren Zeichen" (neben Leerzeichen auch tabs, neue Zeile/carriage return etc.) gematcht werden können, als auch dass die neue Zeile im Ergebnis übernommen werden würde. Dies wurde gelöst, indem alle "leeren Zeichen" zu Leerzeichen konvertiert wurden.

Beispiel (2) wurde erstellt, da es Sonderzeichen enthielt.


\section{Quellcode}
Der Vollständigkeit halber möchte ich hier den Proof-of-concept in bash zeigen, der mit den genannten Sonderfällen nicht funktioniert:
\begin{lstlisting}[language=bash]
gaptext=`cat $1`;
cat $2 | grep -iEo "${gaptext//_/"\w+"}"
\end{lstlisting}

Zur eigentlichen Lösung in Python:
\begin{lstlisting}[language=Python]
def convert_gap_text_to_regex(gap_text):
    """
Converts a gap text to a precompiled regular expression
    """
    re_gap_text = gap_text.strip()
    re_gap_text = re_gap_text.replace(" ","\ ")
    re_gap_text = re_gap_text.replace("_", "\w+")
    return re.compile(re_gap_text)

def tidy_up_source_book(source_book):
    """
Eliminates special Characters and converts any whitespace to a space char
    """
    clean_source = re.sub("[,.:!?;»«_]*", "", source_book)
    clean_source = re.sub("\s+", " ", clean_source)
    clean_source = clean_source.lower()
    return clean_source


def find_all_results(interference, favourite_book):
    """
This function mends the other two into a function that outputs all matches.
    """
    regex = convert_gap_text_to_regex(interference)
    book = tidy_up_source_book(favourite_book)
    return list(set(regex.findall( book)))
\end{lstlisting}


\end{document}
